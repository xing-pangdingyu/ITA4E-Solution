\documentclass[12pt,reqno]{amsart}
\usepackage{amsmath,amsfonts,amscd,amssymb,epsf,color,enumerate,graphicx,url}
\usepackage{algorithm, algorithmic}
\usepackage{forest}
\setlength{\oddsidemargin}{-0.2in}%
\setlength{\evensidemargin}{-0.2in}%
\setlength{\textwidth}{6.6in}%
\setlength{\topmargin}{-0.5in}%
 \setlength{\textheight}{9.5in}%
 \definecolor{orange}{rgb}{1,0.5,0}
 \pagestyle{plain}
\linespread{1.3}
\usepackage[small]{caption}
\newcommand{\pa}{\partial}
\newcommand{\va}{\vspace{0.4cm}}
\newcommand{\di}{\displaystyle}
\newcommand{\disp}{\displaystyle}


% turn on \answertrue to show the solution
% turn on \answerfalse to hide the solution
\newif\ifanswer
\answertrue
%\answerfalse



\begin{document}
\noindent {\footnotesize Introduction to Algorithms}\hspace{10.5cm} {\footnotesize Solutions}

\vspace{0.5cm}
\hspace{5.5cm}\textbf{\large Exercises in Section 6.3}
\vspace{0.5cm}

\begin{enumerate}[1.]

\item Using Figure 6.3 as a model, illustrate the operation of $\textsc{Build-Max-Heap}$ on the array $A=\langle 5, 3, 17, 10, 84, 19, 6, 22, 9 \rangle$.

\ifanswer
\noindent {\bf Solution}
We start from the tree representation of $A$:
\begin{center}
    \begin{forest}
        for tree={
              circle,
              draw,
              fill=lightgray,
              minimum size=1.2cm,
              align=center,
              edge+=-,
              s sep=0.5cm,
              l=1cm
        }
        [$5$
            [$3$
                [$10$, fill=pink
                    [$22$]
                    [$9$]
                ]
                [$84$]
            ]
            [$17$
                [$19$]
                [$6$]
            ]
        ]
    \end{forest}
\end{center}

$\textsc{Max-Heapify}(A, 4)$ exchanges $A[4] = 10$ and $A[8] = 22$:
\begin{center}
    \begin{forest}
        for tree={
              circle,
              draw,
              fill=lightgray,
              minimum size=1.2cm,
              align=center,
              edge+=-,
              s sep=0.5cm,
              l=1cm
        }
        [$5$
            [$3$
                [$22$
                    [$10$]
                    [$9$]
                ]
                [$84$]
            ]
            [$17$, fill=pink
                [$19$]
                [$6$]
            ]
        ]
    \end{forest}
\end{center}

$\textsc{Max-Heapify}(A, 3)$ exchanges $A[3] = 17$ and $A[6] = 19$:
\begin{center}
    \begin{forest}
        for tree={
              circle,
              draw,
              fill=lightgray,
              minimum size=1.2cm,
              align=center,
              edge+=-,
              s sep=0.5cm,
              l=1cm
        }
        [$5$
            [$3$, fill=pink
                [$22$
                    [$10$]
                    [$9$]
                ]
                [$84$]
            ]
            [$19$
                [$17$]
                [$6$]
            ]
        ]
    \end{forest}
\end{center}

$\textsc{Max-Heapify}(A, 2)$ exchanges $A[2] = 3$ and $A[5] = 84$:
\begin{center}
    \begin{forest}
        for tree={
              circle,
              draw,
              fill=lightgray,
              minimum size=1.2cm,
              align=center,
              edge+=-,
              s sep=0.5cm,
              l=1cm
        }
        [$5$, fill=pink
            [$84$
                [$22$
                    [$10$]
                    [$9$]
                ]
                [$3$]
            ]
            [$19$
                [$17$]
                [$6$]
            ]
        ]
    \end{forest}
\end{center}

$\textsc{Max-Heapify}(A, 1)$ exchanges $A[1] = 5$ and $A[2] = 84$, exchanges $A[2] = 5$ and $A[4] = 22$, and exchanges $A[4] = 5$ and $A[8] = 10$:
\begin{center}
    \begin{forest}
        for tree={
              circle,
              draw,
              fill=lightgray,
              minimum size=1.2cm,
              align=center,
              edge+=-,
              s sep=0.5cm,
              l=1cm
        }
        [$84$
            [$22$
                [$10$
                    [$5$]
                    [$9$]
                ]
                [$3$]
            ]
            [$19$
                [$17$]
                [$6$]
            ]
        ]
    \end{forest}
\end{center}
Congratulations! We have built a max-heap from a random array!
\vspace{1cm}



\item Show that $\lceil n/2^{h+1} \rceil \geq 1/2$ for $0\leq h\leq \lfloor\lg{n}\rfloor$.

\ifanswer
\noindent {\bf Solution}
For any $0\leq h\leq \lfloor\lg{n}\rfloor$, $n/2^{h+1}$ is positive. So,
$$
\lceil n/2^{h+1} \rceil \geq 1 \geq 1/2.
$$
\vspace{1cm}



\item Why does the loop index $i$ in line $2$ of $\textsc{Build-Max-Heap}$ decrease from $\lfloor n/2 \rfloor$ to 1 rather than increase from 1 to $\lfloor n/2 \rfloor$?

\ifanswer
\noindent {\bf Solution}
If the loop index $i$ was increasing from 1 to $\lfloor n/2 \rfloor$, $\textsc{Build-Max-Heap}$ may fail to build a max-heap. Let us try to see it from an example. Consider a heap $A = \langle 3, 1, 2, 4 \rangle$, or in tree representation:
\begin{center}
    \begin{forest}
        for tree={
              circle,
              draw,
              fill=lightgray,
              minimum size=1.2cm,
              align=center,
              edge+=-,
              s sep=0.5cm,
              l=1cm
        }
        [$3$
            [$1$
                [$4$]
                [, phantom]
            ]
            [$2$]
        ]
    \end{forest}
\end{center}
If our $i$ was to start from $1$, this program would start by calling $\textsc{Max-Heapify}(A, 1)$, which does nothing to the heap because $A[i]$ is larger than both its children. Then, $\textsc{Max-Heapify}(A, 2)$ would exchange $A[2]$ with $A[4]$, and here is our result:
\begin{center}
    \begin{forest}
        for tree={
              circle,
              draw,
              fill=lightgray,
              minimum size=1.2cm,
              align=center,
              edge+=-,
              s sep=0.5cm,
              l=1cm
        }
        [$3$
            [$4$
                [$1$]
                [, phantom]
            ]
            [$2$]
        ]
    \end{forest}
\end{center}
Did you find the flaw of this procedure? Our program successfully compared $A[1]$ with its children, however, there might be larger elements (especially, the largest one) "hidden" in deeper levels, and a simple $\textsc{Max-Heapify}(A, 1)$ can do nothing about it. Even worse, once we missed the largest element, we could never get a chance to place it to the correct position (that is, the root) as only $\textsc{Max-Heapify}(A, 1)$ has the ability to change the value of $A[1]$! Now, we should pay more attention on the order of our loop indices.
\vspace{1cm}



\item Show that there are at most $\lceil n/2^{h+1} \rceil$ nodes of height $h$ in any $n$-element heap.

\ifanswer
\noindent {\bf Solution}
Let $A$ be an $n$-element heap, we denote $n_h$ as the number of nodes of height $h$. We will prove the proposition desired by induction.\\
The nodes of height 0 are the leaves. By the solution of Exercise 6.1-8, all leaves are indexed by $\lfloor n/2\rfloor+1, \lfloor n/2\rfloor+2, \dots, n$. Therefore,
$$
n_0 = n - \lfloor n/2\rfloor = \lceil n/2\rceil,
$$
which verifies the base case. Now, suppose $n_{h-1} \leq \lceil n/2^{h} \rceil$. We consider two cases:
\begin{enumerate}[(i)]
    \item $n_{h-1}$ is even. Then, each node of height $h$ has exactly $2$ children. Therefore,
    $$
    n_h = n_{h-1}/2 = \lceil n_{h-1}/2 \rceil.
    $$
    \item $n_{h-1}$ is odd. Then, all nodes of height $h$ have $2$ children except exactly one who has only $1$ (left) child. Therefore,
    $$
    n_h = (n_{h-1}+1)/2 = \lceil n_{h-1}/2 \rceil.
    $$
\end{enumerate}
In both cases, we have $n_h = \lceil n_{h-1}/2 \rceil$. Therefore,
$$
n_h \leq \lceil \lceil n/2^{h} \rceil/2 \rceil = \lceil n/2^{h+1} \rceil,
$$
where the last equality holds by equation (3.5).
\vspace{1cm}



\end{enumerate}

\end{document}



