\documentclass[12pt,reqno]{amsart}
\usepackage{amsmath,amsfonts,amscd,amssymb,epsf,color,enumerate,graphicx,url}
\usepackage{forest}
\usepackage{algorithm}
\usepackage{algorithmic}
\setlength{\oddsidemargin}{-0.2in}%
\setlength{\evensidemargin}{-0.2in}%
\setlength{\textwidth}{6.6in}%
\setlength{\topmargin}{-0.5in}%
 \setlength{\textheight}{9.5in}%
 \definecolor{orange}{rgb}{1,0.5,0}
 \pagestyle{plain}
\linespread{1.3}
\usepackage[small]{caption}
\newcommand{\pa}{\partial}
\newcommand{\va}{\vspace{0.4cm}}
\newcommand{\di}{\displaystyle}
\newcommand{\disp}{\displaystyle}


% turn on \answertrue to show the solution
% turn on \answerfalse to hide the solution
\newif\ifanswer
\answertrue
%\answerfalse



\begin{document}
\noindent {\footnotesize Introduction to Algorithms}\hspace{10.5cm} {\footnotesize Solutions}

\vspace{0.5cm}
\hspace{5.5cm}\textbf{\large Exercises in Section 6.2}
\vspace{0.5cm}

\begin{enumerate}[1.]

\item Using Figure 6.2 as a model, illustrate the operation of $\textsc{Max-Heapify}(A, 3)$ on the array $A = \langle 27, 17, 3, 16, 13, 10, 1, 5, 7, 12, 4, 8, 9, 0 \rangle$.

\ifanswer
\noindent {\bf Solution}
We start from the tree representation of $A$:
\begin{center}
\begin{forest}
    for tree={
          circle,
          draw,
          fill=lightgray,
          minimum size=1.2cm,
          align=center,
          edge+=-,
          s sep=0.5cm,
          l=1cm
    }
    [$27$
        [$17$
            [$16$
                [$5$]
                [$7$]
            ]
            [$13$
                [$12$]
                [$4$]
            ]
        ]
        [$3$, fill=pink
            [$10$
                [$8$]
                [$9$]
            ]
            [$1$
                [$0$]
                [, phantom]
            ]
        ]
    ]
\end{forest}
\end{center}

Since $A[3] = 3$, and its children $A[6] = 10$ and $A[7] = 1$, among which $A[6] = 10$ is the maximum, we exchange $A[3]$ with $A[6]$:
\begin{center}
    \begin{forest}
        for tree={
          circle,
          draw,
          fill=lightgray,
          minimum size=1.2cm,
          align=center,
          edge+=-,
          s sep=0.5cm,
          l=1cm
        }
        [$27$
            [$17$
                [$16$
                    [$5$]
                    [$7$]
                ]
                [$13$
                    [$12$]
                    [$4$]
                ]
            ]
            [$10$
                [$3$, fill=pink
                    [$8$]
                    [$9$]
                ]
                [$1$
                    [$0$]
                    [, phantom]
                ]
            ]
        ]
    \end{forest}
\end{center}

Similarly, since $A[6] = 3$, and its children $A[12] = 8$ and $A[13] = 9$, among which $A[13] = 9$ is the maximum, we exchange $A[6]$ with $A[13]$:
\begin{center}
    \begin{forest}
        for tree={
          circle,
          draw,
          fill=lightgray,
          minimum size=1.2cm,
          align=center,
          edge+=-,
          s sep=0.5cm,
          l=1cm
        }
        [$27$
            [$17$
                [$16$
                    [$5$]
                    [$7$]
                ]
                [$13$
                    [$12$]
                    [$4$]
                ]
            ]
            [$10$
                [$9$
                    [$8$]
                    [$3$, fill=pink]
                ]
                [$1$
                    [$0$]
                    [, phantom]
                ]
            ]
        ]
    \end{forest}
\end{center}
Now, we finished the procedure of $\textsc{Max-Heapify}(A, 3)$.
\vspace{1cm}



\item Show that each child of the root of an $n$-node heap is the root of a subtree containing at most $2n/3$ nodes. What is the smallest constant $\alpha$ such that each subtree has at most $\alpha n$ nodes? How does that affect the recurrence (6.1) and its solution?

\ifanswer
\noindent {\bf Solution}
\begin{enumerate}[(i)]
\item Consider the case where the left subtree contains the highest proportion of nodes, that is, this left subtree contains all possible leaves of level $h = \lfloor \lg{n} \rfloor$, while the right subtree contains no leaves of level $h = \lfloor \lg{n} \rfloor$. In tree representation,
\begin{center}
    \begin{forest}
        for tree={
          circle,
          draw,
          fill=lightgray,
          minimum size=0.6cm,
          align=center,
          edge+=-,
          s sep=0.5cm,
          l=1cm
        }
        [,
          [,
            [$\vdots$, draw=none, fill=none
                [,
                    [,]
                    [,]
                ]
                [,
                    [,]
                    [,]
                ]
            ]
            [$\vdots$, draw=none, fill=none
                [,
                    [,]
                    [,]
                ]
                [,
                    [,]
                    [,]
                ]
            ]
          ]
          [,
            [$\vdots$, draw=none, fill=none
                [,]
                [,]
            ]
            [$\vdots$, draw=none, fill=none
                [,]
                [,]
            ]
          ]
        ]
        \node at (7,0.6) {Level};
        \node at (7,0) {0};
        \node at (7,-1.2) {1};
        \node at (7,-2.55) {\vdots};
        \node at (7,-4.1) {h - 1};
        \node at (7,-5.25) {h};
    \end{forest}
\end{center}
The number of nodes of the whole heap is
$$
n = \sum_{i=0}^{h}{2^i} - 2^{h-1} = (2^{h+1} - 1) - 2^{h-1} = 3(2^{h-1} - 1/3).
$$
The number of nodes of the left subtree is
$$
n_l = \sum_{i=1}^{h}{2^{i-1}} = 2^{h} - 1 < 2(2^{h-1} - 1/3) = 2n/3.
$$
\item Every subtree of a heap must have a root other than the root of the heap. Therefore, it must be one of the left and right subtrees we discussed in (i), or a subtree of one of them. Therefore, every subtree has at most $2n/3$ nodes. Hence, $\alpha = 2/3$.
\item The value of $\alpha$ in (ii) tells us why the coefficient of $n$ in recurrence (6.1) is $2/3$ (or, in other words, $b = 3/2$).

\end{enumerate}
\vspace{1cm}



\item Starting with the procedure $\textsc{Max-Heapify}$, write pseudocode for the procedure $\textsc{Min}$-$\textsc{Heapify}(A, i)$, which performs the corresponding manipulation on a min-heap. How does the running time of $\textsc{Min-Heapify}$ compare with that of $\textsc{Max-Heapify}$?

\ifanswer
\noindent {\bf Solution}
\begin{algorithm}
\caption{\textsc{Min-Heapify}$(A, i)$}
\begin{algorithmic}[1]
%\REQUIRE A sorted array $A[1 \dots n]$ and a target value $x$
%\ENSURE Index of $x$ in $A$, or -1 if not found
\STATE $l = \textsc{Left}(i)$
\STATE $r = \textsc{Right}(i)$
\IF{$l \leq A.heap$-$size$ and $A[l] < A[i]$}
    \STATE $smallest = l$
\ELSE
    \STATE $smallest = i$
\ENDIF
\IF{$r \leq A.heap$-$size$ and $A[r] < A[smallest]$}
    \STATE $smallest = r$
\ENDIF
\IF{$smallest \neq i$}
    \STATE exchange $A[i]$ with $A[smallest]$
    \STATE \textsc{Max-Heapify}$(A, smallest)$
\ENDIF
\end{algorithmic}
\end{algorithm}
\\The running time of $\textsc{Min-Heapify}$ is the same as $\textsc{Max-Heapify}$ as they are essentially the same procedure except the order of elements are opposite.
\vspace{1cm}



\item What is the effect of calling $\textsc{Max-Heapify}$ when the element $A[i]$ is larger than its children?

\ifanswer
\noindent {\bf Solution}
Nothing but your computer would still compare the value of $i$ with its children as programmed and decide there is no need to move $A[i]$ at the end.
\vspace{1cm}



\item What is the effect of calling $\textsc{Max-Heapify}$ for $i > A.heap$-$size/2$?

\ifanswer
\noindent {\bf Solution}
For each $i > A.heap$-$size/2$, $i$ does not have a child, that is, $i$ is a leaf. Therefore, the variable $largest$ will be assigned the value $i$, and none of other if statements will be triggered. Therefore, nothing will happen to $A$.
\vspace{1cm}



\item The code for $\textsc{Max-Heapify}$ is quite efficient in terms of constant factors, except possibly for the recursive call in line 10, for which some compilers might produce inefficient code. Write an efficient $\textsc{Max-Heapify}$ that uses an iterative control construct (a loop) instead of recursion.

\ifanswer
\noindent {\bf Solution}
\begin{algorithm}
    \caption{\textsc{Max-Heapify}$(A, i)$}
    \begin{algorithmic}[1]
    \WHILE{True}
        \STATE $l = \textsc{Left}(i)$
        \STATE $r = \textsc{Right}(i)$
        \IF{$l \leq A.heap$-$size$ and $A[l] > A[i]$}
            \STATE $largest = l$
        \ELSE
            \STATE $largest = i$
        \ENDIF
        \IF{$r \leq A.heap$-$size$ and $A[r] > A[largest]$}
            \STATE $largest = r$
        \ENDIF
        \IF{$largest = i$}
            \STATE \textbf{break}
        \ENDIF
        \STATE exchange $A[i]$ with $A[largest]$
        \STATE $i = largest$
    \ENDWHILE
    \end{algorithmic}
\end{algorithm}
\vspace{1cm}



\item Show that the worst-case running time of $\textsc{Max-Heapify}$ on a heap of size $n$ is $\Omega(\lg{n})$. (\textit{Hint:} For a heap with $n$ nodes, give node values that cause $\textsc{Max-Heapify}$ to be called recursively at every node on a simple path from the root down to a leaf.)

\ifanswer
\noindent {\bf Solution}
Exercise 6.1-2 tells us the height of a heap with $n$ nodes is $h = \lfloor \lg{n} \rfloor$. Consider the worst case where, in particular, the left most path consists of the $h+1$ largest elements that are in descending order (as shown in the graph below, assuming $a < a_0 < a_1 < \cdots < a_h$, where $a$ stands for any other elements in this heap).
\begin{center}
    \begin{forest}
        for tree={
          circle,
          draw,
          fill=lightgray,
          minimum size=1.2cm,
          align=center,
          edge+=-,
          s sep=0.3cm,
          l=1cm
        }
        [$a_0$
          [$a_1$
            [$\vdots$, draw=none, fill=none
                [$a_{h-1}$
                    [$a_h$]
                    [,]
                ]
                [,
                    [,]
                    [,]
                ]
            ]
            [$\vdots$, draw=none, fill=none
                [,
                    [,]
                    [, phantom]
                ]
                [,]
            ]
          ]
          [,
            [$\vdots$, draw=none, fill=none
                [,]
                [,]
            ]
            [$\vdots$, draw=none, fill=none
                [,]
                [,]
            ]
          ]
        ]
    \end{forest}
\end{center}
Therefore, if we call $\textsc{Max-Heapify}(A, 1)$, $a_1$ must be the largest element among $a_0$ and its children, and $\textsc{Max-Heapify}(A, 2)$ will be called as an iteration. Again, $a_2$ must be the largest among $a_1$ and its children, so $\textsc{Max-Heapify}(A, 4)$ will be called as an iteration. Repeat this process, the following iterations are $\textsc{Max-Heapify}(A, 8)$, $\textsc{Max-Heapify}(A, 16)$, $\dots$, $\textsc{Max-Heapify}(A, 2^{h-1})$, with at least $h = \lfloor \lg{n} \rfloor$ iterations in total, and with running time $\Theta(1)$ for each. Hence, the running time of $\textsc{Max-Heapify}$ is
$$
\Omega{\left(\lfloor\lg{n}\rfloor\Theta(1)\right)} = \Omega(\lfloor \lg{n} \rfloor) = \Omega(\lg{n}).
$$
\vspace{1cm}



\end{enumerate}

\end{document}



