\documentclass[12pt,reqno]{amsart}
\usepackage{amsmath,amsfonts,amscd,amssymb,epsf,color,enumerate,graphicx,url}
\usepackage{forest}
\setlength{\oddsidemargin}{-0.2in}%
\setlength{\evensidemargin}{-0.2in}%
\setlength{\textwidth}{6.6in}%
\setlength{\topmargin}{-0.5in}%
 \setlength{\textheight}{9.5in}%
 \definecolor{orange}{rgb}{1,0.5,0}
 \pagestyle{plain}
\linespread{1.3}
\usepackage[small]{caption}
\newcommand{\pa}{\partial}
\newcommand{\va}{\vspace{0.4cm}}
\newcommand{\di}{\displaystyle}
\newcommand{\disp}{\displaystyle}


% turn on \answertrue to show the solution
% turn on \answerfalse to hide the solution
\newif\ifanswer
\answertrue
%\answerfalse



\begin{document}
\noindent {\footnotesize Introduction to Algorithms}\hspace{10.5cm} {\footnotesize Solutions}

\vspace{0.5cm}
\hspace{5.5cm}\textbf{\large Exercises in Section 6.1}
\vspace{0.5cm}

\begin{enumerate}[1.]

\item What are the minimum and maximum numbers of elements in a heap of height $h$?

\ifanswer
\noindent {\bf Solution}
\begin{enumerate}[(i)]
\item Minimum: There is only one node at level $h$.
\begin{center}
\begin{forest}
    for tree={
      circle,
      draw,
      fill=lightgray,
      minimum size=0.6cm,
      align=center,
      edge+=-,
      s sep=0.5cm,
      l=1cm
    }
    [,
        [,
            [\vdots, draw=none, fill=none
                [,
                    [,]
                    [, phantom]
                ]
                [,]
            ]
            [\vdots, draw=none, fill=none
                [,]
                [,]
            ]
        ]
        [,
            [\vdots, draw=none, fill=none
                [,]
                [,]
            ]
            [\vdots, draw=none, fill=none
                [,]
                [,]
            ]
        ]
    ]
    \node at (7,0.6) {Level};
    \node at (7,0) {0};
    \node at (7,-1.2) {1};
    \node at (7,-2.55) {\vdots};
    \node at (7,-4.1) {h - 1};
    \node at (7,-5.25) {h};
\end{forest}
\end{center}
$$
\textit{Minimum number of elements} = 1 + \sum_{i = 0}^{h-1}{2^i} = 2^h.
$$
\item Maximum: There are $2^h$ nodes, which is the maximally possible number, at level $h$.
\begin{center}
    \begin{forest}
        for tree={
          circle,
          draw,
          fill=lightgray,
          minimum size=0.6cm,
          align=center,
          edge+=-,
          s sep=0.5cm,
          l=1cm
        }
        [,
            [,
                [\vdots, draw=none, fill=none
                    [,]
                    [,]
                ]
                [\vdots, draw=none, fill=none
                    [,]
                    [,]
                ]
            ]
            [,
                [\vdots, draw=none, fill=none
                    [,]
                    [,]
                ]
                [\vdots, draw=none, fill=none
                    [,]
                    [,]
                ]
            ]
        ]
        \node at (7,0.6) {Level};
        \node at (7,0) {0};
        \node at (7,-1.2) {1};
        \node at (7,-2.55) {\vdots};
        \node at (7,-4.1) {h};
\end{forest}
\end{center}
\vspace{0.5cm}
$$
\textit{Maximum number of elements} = \sum_{i = 0}^{h}{2^i} = 2^{h+1}-1.
$$



\end{enumerate}
\vspace{1cm}



\item Show that an $n$-element heap has height $\lfloor\lg{n}\rfloor$.

\ifanswer
\noindent {\bf Solution}
In Exercise 6.1-1, we have found a set $\{2^h, 2^h+1, ..., 2^{h+1}-1\}$ consisting of all possible number of elements in a heap of height $h$. Notice that these sets (for each nonnegative integer $h$) cover the set of positive integers, that is,
$$
\sum_{h=0}^{\infty}{\{2^h, 2^h+1, ..., 2^{h+1}-1\}} = \mathbb{Z}^{+}.
$$
Also, all these sets are disjoint, that is, do not coincide with each other. Therefore, for any given $n$-element heap, there must be a unique $h$ such that
$$
2^h \leq n \leq 2^{h+1} - 1.
$$
Take floor of logarithm on all parts of this inequality, we get
$$
h = \lfloor \lg{2^h} \rfloor \leq \lfloor \lg{n} \rfloor \leq \lfloor \lg{(2^{h+1}-1)} \rfloor = h,
$$
where the last equality holds because
$$
h = \lg{2^h} \leq \lg{(2^{h+1}-1)} < \lg{2^{h+1}} = h+1.
$$
Hence, $h = \lfloor \lg{n} \rfloor$.
\vspace{1cm}



\item Show that in any subtree of a max-heap, the root of the subtree contains the largest value occurring anywhere in that subtree.

\ifanswer
\noindent {\bf Solution}
Note that a subtree of a max-heap is itself a max-heap. Suppose the largest value is not contained in an element other than the root, say $e$. Then, $\textup{Parent}(e)$, which represents the same element in both the subtree and the original max-heap, must have a smaller value than $e$, which contradicts the definition of max-heaps.
\vspace{1cm}



\item Where in a max-heap might the smallest element reside, assuming that all elements are distinct?

\ifanswer
\noindent {\bf Solution}
By the definition of max-heaps, the smallest element should have no children. Therefore, the smallest element resides only on the leaves of the max-heap.
\vspace{1cm}



\item At which levels in a max-heap might the $k$-th largest element reside, for $2\leq k\leq \lfloor n/2\rfloor$, assuming that all elements are distinct?

\ifanswer
\noindent {\bf Solution}
First, we find the upper bound of the levels a $k$-th largest element may reside. Exercise 6.1-2 gives a trivial bound $h = \lfloor \lg{n} \rfloor$, the height of an $n$-element heap, and of course, an element cannot have a level bigger than the height of the heap. Another trivial upper bound is $k-1$, because there are at most $k-1$ elements in the path from the root to the $k$-th largest element. Therefore, we get a upper bound $\textup{min}\{\lfloor \lg{n} \rfloor, k-1\}$. To achieve this upper bound, we put the $2$-nd largest element as the left child to the root, the $3$-rd largest element as the left child to the $2$-nd largest element, and repeat this procedure until the level reaches $\textup{min}\{\lfloor \lg{n} \rfloor, k-1\}$, where we put the $k$-th largest element.\\
\indent The lower bound is trivial too, and it is $1$. It is easy to see this bound is tight, because we can always put the $k$-th largest element as a child of the root, and since $k\leq\lfloor n/2 \rfloor$, there are at least $\lceil n/2 \rceil$ elements that are smaller than the $k$-th largest element, and we can safely put the smallest $\lceil n/2 \rceil$ elements under the $k$-th largest element as its descendants (the elements in the subtree induced by an element, except the element itself).\\
\indent Hence, the levels in a max-heap the $k$-th largest element may reside are in the set $\{1, 2, \dots, \textup{min}\{\lfloor \lg{n} \rfloor, k-1\}\}$.
\vspace{1cm}



\item Is an array that is in sorted order a min-heap?

\ifanswer
\noindent {\bf Solution}
Yes. Let $A$ be an array that is sorted in ascending order. $A[1]$ is the root, and for each $i=2, 3, \dots, n$, we have
$$
\textup{Parent}(i) = \lfloor\frac{i}{2}\rfloor < \frac{i}{2}+1 \leq \frac{i}{2}+1 + (\frac{i}{2} - 1) = i.
$$
Since A is sorted in ascending order,
$$
A[\textup{Parent}(i)] < A(i).
$$
Therefore, A is a min-heap.
\vspace{1cm}



\item Is the array with values $\langle 33, 19, 20, 15, 13, 10, 2, 13, 16, 12 \rangle$ a max-heap?

\ifanswer
\noindent {\bf Solution}
No, because
$$
A[\textup{Parent}(9)] = A[4] = 15 < 16 = A[9].
$$
This contradicts the definition of max-heaps.
\vspace{1cm}



\item Show that, with the array representation for storing an $n$-element heap, the leaves are the nodes indexed by $\lfloor n/2\rfloor+1, \lfloor n/2\rfloor+2, \dots, n$.

\ifanswer
\noindent {\bf Solution}
For each node of index $i = 1, 2, \dots, \lfloor n/2\rfloor$, it has a left child indexed $2i \leq 2\lfloor n/2\rfloor \leq n$. For each node of index $j = \lfloor n/2\rfloor+1, \lfloor n/2\rfloor+2, \dots, n$, we have
$$
2j \geq 2(\lfloor \frac{n}{2}\rfloor + 1) > 2\left((\frac{n}{2} - 1) + 1\right) = n.
$$
So, they should not have a left child, and therefore, they are leaves.
\vspace{1cm}



\end{enumerate}

\end{document}



