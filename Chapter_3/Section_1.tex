\documentclass[12pt,reqno]{amsart}
%\documentclass[../Solutions_Introduction_to_Algorithms.tex]{subfiles}
\usepackage{amsmath,amsfonts,amscd,amssymb,epsf,color,enumerate,graphicx,url}
\usepackage{algorithm, algorithmic}
\usepackage{forest, tikz, xcolor}
\usepackage{parskip}
\usetikzlibrary{matrix, positioning}
\usetikzlibrary{positioning,arrows.meta}
\setlength{\oddsidemargin}{-0.2in}%
\setlength{\evensidemargin}{-0.2in}%
\setlength{\textwidth}{6.6in}%
\setlength{\topmargin}{-0.5in}%
 \setlength{\textheight}{9.5in}%
 \definecolor{orange}{rgb}{1,0.5,0}
 \pagestyle{plain}
\linespread{1.3}
\usepackage[small]{caption}
\newcommand{\pa}{\partial}
\newcommand{\va}{\vspace{0.4cm}}
\newcommand{\di}{\displaystyle}
\newcommand{\disp}{\displaystyle}


% turn on \answertrue to show the solution
% turn on \answerfalse to hide the solution
\newif\ifanswer
\answertrue
%\answerfalse



\begin{document}
\noindent {\footnotesize Introduction to Algorithms}\hspace{10.5cm} {\footnotesize Solutions}

\vspace{0.5cm}
\hspace{5.5cm}\textbf{\large Exercises in Section 3.1}
\vspace{0.5cm}

\begin{enumerate}[1.]

\item Modify the lower-bound argument for insertion sort to handle input sizes that are not necessarily a multiple of $3$.
\vspace{0.5cm}

\ifanswer
\noindent {\bf Solution}

Based on Figure~3.1,
\begin{itemize}
    \item For $n = 3k$, each of the $k$ largest values moves through each of these $k$ positions to somewhere in these $k$ positions: $(n/3)(n/3) = n^2/9 = \Omega(n^2)$.
    \item For $n = 3k + 1$, each of the $k$ largest values moves through each of these $k + 1$ positions to somewhere in these $k$ positions: $((n-1)/3)((n+2)/3) = n^2/9 + n/9 - 2/9 = \Omega(n^2)$.
    \item For $n = 3k + 2$, each of the $k$ largest values moves through each of these $k + 2$ positions to somewhere in these $k$ positions: $((n-2)/3)((n+1)/3) = n^2/9 - n/9 - 2/9 = \Omega(n^2)$.
\end{itemize}

\vspace{1cm}



\item Using reasoning similar to what we used for insertion sort, analyze the running time of the selection sort algorithm from Exercise~2.2-2.

\vspace{0.5cm}

\ifanswer
\noindent {\bf Solution}

The two \textbf{for} loops must be executed $(n-1)(n-1) = \Theta(n^2)$ times in any case. Therefore the running time of the selection sort is $\Theta(n^2)$.

\vspace{1cm}



\item Suppose that $\alpha$ is a fraction in the range $0 < \alpha < 1$. Show how to generalize
the lower-bound argument for insertion sort to consider an input in which the $\alpha$
largest values start in the first $\alpha n$ positions. What additional restriction do you
need to put on $\alpha$? What value of $\alpha$ maximizes the number of times that the
$\alpha$ largest values must pass through each of the middle $(1 - 2\alpha)n$ array positions?

\vspace{0.5cm}

\ifanswer
\noindent {\bf Solution}

Additional restriction: $\alpha < 1/2$.

To maximize $T(n) = \alpha(1-2\alpha)n^2$, find the partial derivative: $$\frac{\partial T}{\partial\alpha} = (1-4\alpha)n^2,$$ which equals zero when $\alpha = 1/4$.

\vspace{1cm}






\end{enumerate}

\end{document}



