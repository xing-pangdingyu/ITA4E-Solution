\documentclass[12pt,reqno]{amsart}
%\documentclass[../Solutions_Introduction_to_Algorithms.tex]{subfiles}
\usepackage{amsmath,amsfonts,amscd,amssymb,epsf,color,enumerate,graphicx,url}
\usepackage{algorithm, algorithmic}
\usepackage{forest, tikz, xcolor}
\usepackage{parskip}
\usetikzlibrary{matrix, positioning}
\usetikzlibrary{positioning,arrows.meta}
\setlength{\oddsidemargin}{-0.2in}%
\setlength{\evensidemargin}{-0.2in}%
\setlength{\textwidth}{6.6in}%
\setlength{\topmargin}{-0.5in}%
 \setlength{\textheight}{9.5in}%
 \definecolor{orange}{rgb}{1,0.5,0}
 \pagestyle{plain}
\linespread{1.3}
\usepackage[small]{caption}
\newcommand{\pa}{\partial}
\newcommand{\va}{\vspace{0.4cm}}
\newcommand{\di}{\displaystyle}
\newcommand{\disp}{\displaystyle}


% turn on \answertrue to show the solution
% turn on \answerfalse to hide the solution
\newif\ifanswer
\answertrue
%\answerfalse



\begin{document}
\noindent {\footnotesize Introduction to Algorithms}\hspace{10.5cm} {\footnotesize Solutions}

\vspace{0.5cm}
\hspace{5.5cm}\textbf{\large Exercises in Section 4.1}
\vspace{0.5cm}

\begin{enumerate}[1.]

\item Generalize \textsc{Matrix-Multiply-Recursive} to multiply $n \times n$ matrices for which
$n$ is not necessarily an exact power of $2$. Give a recurrence describing its running
time. Argue that it runs in $\Theta(n^3)$ time in the worst case.

\vspace{0.5cm}

\ifanswer
\noindent {\bf Solution}

\begin{algorithm}
    \caption{\textsc{Matrix-Multiply-Recursive-Modified}$(A, B, C, n)$}
    \begin{algorithmic}[1]
        \STATE find the smallest integer $k$ such that $n \leq 2^k$
        \IF{$n \neq 2^k$}
            \STATE create $2^k \times 2^k$ matrices $A'$ and $B'$, whose top-left
            \STATE $n \times n$ submatrices are copies of $A$ and $B$, respectively;
            \STATE and whose other entries are set to $0$.
            \RETURN
        \ELSE
            \STATE $A' = A$ and $B' = B$
        \ENDIF
        \STATE \textsc{Matrix-Multiply-Recursive}$(A', B', C, n)$
    \end{algorithmic}
\end{algorithm}

Denote $T(n)$ as the running time of \textsc{Matrix-Multiply-Recursive}$(A, B, C, n)$, the running time $T'(n)$ of \textsc{Matrix-Multiply-Recursive-Modified}$(A, B, C, n)$ is between $T(n)$ and $T(2n)$. Since $$T(2n) = \Theta((2n)^3) = \Theta(n^3) = T(n),$$ we conclude that $T'(n) = \Theta(n^3)$.

\vspace{1cm}



\item How quickly can you multiply a $kn \times n$ matrix ($kn$ rows and $n$ columns) by an
$n \times kn$ matrix, where $k \geq 1$, using \textsc{Matrix-Multiply-Recursive} as a subroutine?
Answer the same question for multiplying an $n \times kn$ matrix by a $kn \times n$
matrix. Which is asymptotically faster, and by how much?

\vspace{0.5cm}

\ifanswer
\noindent {\bf Solution}

Consider the case $n = 3$. Let $A$ be a $kn \times n$ matrix, and $B$ be a $n \times kn$ matrix. Then, divide $A$ and $B$ into $k$ submatrices of size $n \times n$. Multiply $A$ by $B$:
$$
\begin{bmatrix}
A_1 \\[6pt]
A_2 \\[6pt]
A_3
\end{bmatrix}
\;
\begin{bmatrix}
B_1 & B_2 & B_3
\end{bmatrix}
=
\begin{bmatrix}
A_1B_1 & A_1B_2 & A_1B_3 \\[6pt]
A_2B_1 & A_2B_2 & A_2B_3 \\[6pt]
A_3B_1 & A_3B_2 & A_3B_3
\end{bmatrix}.
$$
As we can see, there requires $k^2$ multiplications, and each of them can be solved in $\Theta(n^3)$ time. Therefore, the total running time is $\Theta(k^2n^3)$.

When multiplying an $n \times kn$ matrix by a $kn \times n$ matrix, we would have
$$
\begin{bmatrix}
A_1 & A_2 & A_3
\end{bmatrix}
\;
\begin{bmatrix}
B_1 \\[6pt]
B_2 \\[6pt]
B_3
\end{bmatrix}
=
\begin{bmatrix}
A_1B_1 + A_2B_2 +A_3B_3
\end{bmatrix}.
$$
In this case, there requires only $k$ multiplications. Therefore, the total running time is $\Theta(kn^3)$.

\vspace{1cm}



\item Suppose that instead of partitioning matrices by index calculation in \textsc{Matrix-Multiply-Recursive},
you copy the appropriate elements of $A$, $B$, and $C$ into separate $\dfrac{n}{2}\times\dfrac{n}{2}$
submatrices $A_{11},A_{12},A_{21},A_{22}$; $B_{11},B_{12},B_{21},B_{22}$; and
$C_{11},C_{12},C_{21},C_{22}$, respectively. After the recursive calls, you copy the results
from $C_{11},C_{12},C_{21},C_{22}$ back into the appropriate places in $C$. How does
recurrence (4.9) change, and what is its solution?

\vspace{0.5cm}

\ifanswer
\noindent {\bf Solution}

If so, the partitioning time will become $\Theta(n^2)$ instead of $\Theta(1)$. The recurrence (4.9) would be $$T(n) = 8T(n/2) + \Theta(n^2).$$ Using the master method, we have the solution remaining $T(n) = \Theta(n^3)$.

\vspace{1cm}



\item Write pseudocode for a divide-and-conquer algorithm \textsc{Matrix-Add-Recursive}
that sums two $n\times n$ matrices $A$ and $B$ by partitioning each of them into four
$\dfrac{n}{2}\times\dfrac{n}{2}$ submatrices and then recursively summing corresponding
pairs of submatrices. Assume that matrix partitioning uses $\Theta(1)$-time index
calculations. Write a recurrence for the worst-case running time of
\textsc{Matrix-Add-Recursive}, and solve your recurrence. What happens if you use
$\Theta(n^2)$-time copying to implement the partitioning instead of index calculations?

\vspace{0.5cm}

\ifanswer
\noindent {\bf Solution}

\begin{itemize}
\item Recurrence using index calculations: $$T(n) = 4T(n/2) + \Theta(1).$$ Using the master method, the solution is $T(n) = \Theta(n^2)$.
\item Recurrence using copying: $$T(n) = 4T(n/2) + \Theta(n^2).$$ Using the master method, the solution is $T(n) = \Theta(n^2\lg n)$.
\end{itemize}

\vspace{1cm}





\end{enumerate}

\end{document}



